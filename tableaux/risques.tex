\begin{tabularx}{\linewidth}{
    |>{\hsize=0.40\hsize}X|% 10% of 4\hsize 
    >{\hsize=0.25\hsize}X|% 30% of 4\hsize
    >{\hsize=0.25\hsize}X|% 30% of 4\hsize 
    >{\hsize=0.1\hsize}X|% 30% of 4\hsize 
       % sum=0.2\hsize for 4 columns
  }
    \hline
    \textbf{Risque} & \textbf{Mitigation} & \textbf{Conséquence} & \textbf{Priorité}\\\hline
    \st{Soucis de compatibilités avec l'application mobile} & \st{Utilisation d'un framework morderne} & \st{L'application mobile pourrait ne pas foncitonner sur tous les cellulaires, particulièrement ios} & 4 \\\hline %Android ftw
    Problème de "fantôme" dans le Epaper & Avoir un écran de remplacement & Pixels perdus temporairement ou à jamais & 1 \\\hline %2022-03-06 
    % encore à jour? Je pense ils était là la semaine passé -> oui 
    %Commumication entre Info, Elec, Mec, Robo & Prendre du temps pour parler au lieu d'utiliser des post & Désorganisation de l'équipe & 2\\\hline
    Oublier une spécification du rulebook puisque celui-ci à changer & Faire une revue du rulebook et ajouter/enlever les requis dans le DVP & Disqualifier de la compétition avant même d'avoir un prototype & 1\\\hline
    
    Ne pas savoir contrôler un moteur adéquatement & Allez chercher l'expertise requise & Amélioration de l'effacacité ou durée de vie potentiel & 4 \\\hline
    \st{Ne pas avoir du code fonctionnel sur la branche principale & Surveiller et tester les "pull requests" & Besoin de mettre du temps pour récupérer une version fonctionnelle &  2 \\\hline
    Ne pas avoir de schéma électrique du circuit de contrôle de l'instrumentation & Faire le suivi avec Joel et Vincent & Retard sur l'objectif d'intégration prévu pour la fin de session & 2 \\\hline % 2022-03-21
  \end{tabularx}
  
  
% Template pour le tableau des risques de la semaine : 

% Risque : le risque de la semaine
% Mitigation: comment mitiger le risque pour réduire ses conséquences/occurances/etc.
% Conséquence : La ou les conséquences si le risque survient
% Priorité : Priorité du risque sur une échelle de 1 à 5.  5 étant + prioritaire. La priorité est basé sur les conséquences la probablité d'occurence, etc.  
  
%  \textbf{Risque} & \textbf{Mitigation} & \textbf{Conséquence} & \textbf{Priorité}\\\hline
%    Risque & Mitigation & Conséquence & Priorité\\\hline
%    Risque & Mitigation & Conséquence & Priorité\\\hline
%    Risque & Mitigation & Conséquence & Priorité\\\hline
%    Risque & Mitigation & Conséquence & Priorité\\\hline
%    Risque & Mitigation & Conséquence & Priorité\\\hline
%    Risque & Mitigation & Conséquence & Priorité\\\hline
%    Risque & Mitigation & Conséquence & Priorité\\\hline
%    Risque & Mitigation & Conséquence & Priorité\\\hline
%    Risque & Mitigation & Conséquence & Priorité\\\hline
%    Risque & Mitigation & Conséquence & Priorité\\\hline
%    Risque & Mitigation & Conséquence & Priorité\\\hline
%    Risque & Mitigation & Conséquence & Priorité\\\hline