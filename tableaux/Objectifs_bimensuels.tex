\textbf{\large Contrôle Moteur / BMS}\\
\begin{tabularx}{\linewidth}{
    |>{\hsize=0.33\hsize}X|
    >{\hsize=0.33\hsize}X|
    >{\hsize=0.33\hsize}X|
    >{\hsize=2.5\hsize}X|% 10% of 4\hsize 
    >{\hsize=0.5\hsize}X|% 30% of 4\hsize
       % sum=0.2\hsize for 4 columns
  }
    \hline
    \textbf{Planifié} & \textbf{Progrès} & \textbf{Heures} &\textbf{Objectif} & \textbf{Responsable} \\\hline
10\% & 10\% & & Implémenter un filtre passe-bas logiciel pour les lectures de courant & Gabriel Q.\\\hline
      10\% & 10\% &  & Intégrer la lecture de tension des capteurs de courant  & Louis Tardif.\\\hline
      50\% & 50\% &  & Intégrer la lecture de la température à partir des capteurs de l'onduleur & Louis Tardif.\\\hline 
\end{tabularx}
\newline

\hfill \break
\textbf{\large Simulateur}
\\
\begin{tabularx}{\linewidth}{
    |>{\hsize=0.33\hsize}X|
    >{\hsize=0.33\hsize}X|
    >{\hsize=0.33\hsize}X|
    >{\hsize=2.5\hsize}X|% 10% of 4\hsize 
    >{\hsize=0.5\hsize}X|% 30% of 4\hsize
       % sum=0.2\hsize for 4 columns
  }
    \hline
    \textbf{Planifié} & \textbf{Progrès} & \textbf{Heures} &\textbf{Objectif} & \textbf{Responsable} \\\hline
        50\% & 25\% &  &  Recherche priliminaire pour data dymo & Malik C.\\\hline
        0 \% & 0\% &  &  Intégration du parcours sur le dyno de Pascal Messier. & Malik C.\\\hline % nouveau stuff
        100\% & 100\% &  &  Planifier les tâches du simulateur & Claude G-P.\\\hline % Claude 2022-09-07
        100\% & 100\% &  &  Fix bug pour lancer la simulation via le UI& Claude G.-P.\\\hline % Claude 2022-09-07
        50\% & 75\% &  &  Rendre la modification des paramêtres plus intuitif & Mathieu P.\\\hline % Claude 2022-09-07
        0\% & 0\% &  &  Ajouter des arrêts au pilote (stratégie de conduite) & Claude G.-P.\\\hline % Claude 2022-09-07

\end{tabularx}\\

\hfill \break
\textbf{\large Instrumentation}\\
\begin{tabularx}{\linewidth}{
    |>{\hsize=0.33\hsize}X|
    >{\hsize=0.33\hsize}X|
    >{\hsize=0.33\hsize}X|
    >{\hsize=2.5\hsize}X|% 10% of 4\hsize 
    >{\hsize=0.5\hsize}X|% 30% of 4\hsize
       % sum=0.2\hsize for 4 columns
  }
    \hline
    \textbf{Planifié} & \textbf{Progrès} & \textbf{Heures} &\textbf{Objectif} & \textbf{Responsable} \\\hline
     0 \% & 0\% &  &   Montage prototype avec PCB d'instrumentation & Alexandre B. \\\hline 
     0 \% & 0\% &  &   Plan du cablage d'instrumentation & Alexandre B. \\\hline 
     50 \% & 30\% & 18 &   Validation de la librairie d'affichage sur le E-Paper & Charles-E. G. \\\hline 
     20 \% & 50\% &  &   Communication entre l'application mobile et le simulateur (MQTT) & Marian L.R. \\\hline 
     30 \% & 50\% &  &   Mise à jour du code de l'essuie glace pour utiliser un servo moteur. & Alexandre Bergeron \\\hline 
     
\end{tabularx}

% TEMPLATE des lignes du tableau de taches accomplis cette semaine
%https://www.overleaf.com/project/60dbafb3c22aac53e265b6e6
% Tâche: résumé de la tâche
% Système : Numéro de système ou nom complet ex : Simulateur ou SIM1
% Responsable : Initiales ou nom complet ex : Gabriel Cabana ou G.C.
% Heures : Heures passées par le responsable à faire la tâche la semaine dernière (jeudi à jeudi)
%
%\textbf{Tâche} & \textbf{Système} & \textbf{Responsable} & \textbf{Heures}\\\hline
%    Tâche & Système & Responsable & Heures\\\hline
%    Tâche & Système & Responsable & Heures\\\hline
%    Tâche & Système & Responsable & Heures\\\hline
%    Tâche & Système & Responsable & Heures\\\hline
%    Tâche & Système & Responsable & Heures\\\hline
%    Tâche & Système & Responsable & Heures\\\hline
%    Tâche & Système & Responsable & Heures\\\hline
%    Tâche & Système & Responsable & Heures\\\hline
%    Tâche & Système & Responsable & Heures\\\hline
%    Tâche & Système & Responsable & Heures\\\hline
%    Tâche & Système & Responsable & Heures\\\hline
%    Tâche & Système & Responsable & Heures\\\hline
%  