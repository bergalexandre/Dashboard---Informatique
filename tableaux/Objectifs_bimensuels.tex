\textbf{\large Contrôle Moteur / BMS}\\
\begin{tabularx}{\linewidth}{
    |>{\hsize=0.33\hsize}X|
    >{\hsize=0.33\hsize}X|
    >{\hsize=0.33\hsize}X|
    >{\hsize=2.5\hsize}X|% 10% of 4\hsize 
    >{\hsize=0.5\hsize}X|% 30% of 4\hsize
       % sum=0.2\hsize for 4 columns
  }
    \hline
    \textbf{Planifié(\%)} & \textbf{Progrès(\%)} & \textbf{Planifié(H)} &\textbf{Objectif} & \textbf{Responsable} \\\hline
    50\% & 50\% & 15 & Synchroniser les lectures d'ADC & Louis T.\\\hline
    60\% & 60\% & 2 & Implémenter PID de vitesse & Gabriel Q.\\\hline
    50\% & 50\% & 10 & Lire les températures du l'onduleur et du moteur & Gabriel Q.\\\hline
    50\% & 50\% & 5 & Compenser l'angle du moteur & Gabriel Q.\\\hline
\end{tabularx}
\newline

\hfill \break
\textbf{\large Simulateur}
\\
\begin{tabularx}{\linewidth}{
    |>{\hsize=0.33\hsize}X|
    >{\hsize=0.33\hsize}X|
    >{\hsize=0.33\hsize}X|
    >{\hsize=2.5\hsize}X|% 10% of 4\hsize 
    >{\hsize=0.5\hsize}X|% 30% of 4\hsize 
       % sum=0.2\hsize for 4 columns 
  }
    \hline
    \textbf{Planifié(\%)} & \textbf{Progrès(\%)} & \textbf{Planifié(H)} &\textbf{Objectif} & \textbf{Responsable} \\\hline
        10 \% & 10\% & 18 &  Ajouter page de télémétrie dans l'interface graphique. & Malik C. \\\hline
        95\% & 90\% & 16 &  Décélération/acélération dans les virages (stratégie de conduite) & Claude G.P. \\\hline % 2022-11-09
        100\% & 90\% & 30 &  Conception et intégration contrôleur de vitesse (machine à état/logique floue)  & Mathieu P. \\\hline % ?
        0\% & 0\% & 8 &  Mettre à jour notre modèle mathématique (28 novembre) & Claude G.P. \\\hline % 2022-10-19
        100\% & 100\% & 16 &  \st{faire du CSS pour polir l'interface} & Claude G.P. \\\hline % 2022-10-19
\end{tabularx}\\

\hfill \break
\textbf{\large Instrumentation}\\
\begin{tabularx}{\linewidth}{
    |>{\hsize=0.33\hsize}X|
    >{\hsize=0.33\hsize}X|
    >{\hsize=0.33\hsize}X|
    >{\hsize=2.5\hsize}X|% 10% of 4\hsize 
    >{\hsize=0.5\hsize}X|% 30% of 4\hsize
       % sum=0.2\hsize for 4 columns
  }
    \hline
    \textbf{Planifié(\%)} & \textbf{Progrès(\%)} & \textbf{Planifié(H)} &\textbf{Objectif} & \textbf{Responsable} \\\hline
     100 \% & 100\% & 9 \rightarrow18 &  \st{Contrôler les lumières du véhicule} & Charles-E. G. \\\hline
     50 \% & 50\% & 9 &  Ajout d'erreurs sur le EPAPER & Charles-E. G. \\\hline
     50 \% & 50\% & 8 & Configuration et test du chargeur & William R. et Marian L.R. \\\hline
     0 \% & 0\% & 8 & Recuillir les données du BMS (Idneo) & William R. et Marian L.R. \\\hline
     100 \% & 100\% & 24 & \st{Test de la carte prototype} & Alexandre B. et Charles-E. G. \\\hline
     100\% & 95\% & 24 & Intégration de la carte prototype au véhicule & Alexandre B. \\\hline
     60\% & 40\% & 30 & Traîter les donnée CAN dans instrumentation (envoie via bluetooth et affichage sur EPaper). & Alexandre B. \\\hline
\end{tabularx}\\

% TEMPLATE des lignes du tableau de taches accomplis cette semaine
%https://www.overleaf.com/project/60dbafb3c22aac53e265b6e6
% Tâche: résumé de la tâche
% Système : Numéro de système ou nom complet ex : Simulateur ou SIM1
% Responsable : Initiales ou nom complet ex : Gabriel Cabana ou G.C.
% Heures : Heures passées par le responsable à faire la tâche la semaine dernière (jeudi à jeudi)
%
%\textbf{Tâche} & \textbf{Système} & \textbf{Responsable} & \textbf{Heures}\\\hline
%    Tâche & Système & Responsable & Heures\\\hline
%    Tâche & Système & Responsable & Heures\\\hline
%    Tâche & Système & Responsable & Heures\\\hline
%    Tâche & Système & Responsable & Heures\\\hline
%    Tâche & Système & Responsable & Heures\\\hline
%    Tâche & Système & Responsable & Heures\\\hline
%    Tâche & Système & Responsable & Heures\\\hline
%    Tâche & Système & Responsable & Heures\\\hline
%    Tâche & Système & Responsable & Heures\\\hline
%    Tâche & Système & Responsable & Heures\\\hline
%    Tâche & Système & Responsable & Heures\\\hline
%    Tâche & Système & Responsable & Heures\\\hline
%  