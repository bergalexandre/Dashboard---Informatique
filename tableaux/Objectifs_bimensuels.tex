\textbf{\large Contrôle Moteur / BMS}\\
\begin{tabularx}{\linewidth}{
    |>{\hsize=0.5\hsize}X|
    >{\hsize=0.5\hsize}X|
    >{\hsize=2.5\hsize}X|% 10% of 4\hsize 
    >{\hsize=0.5\hsize}X|% 30% of 4\hsize
       % sum=0.2\hsize for 4 columns
  }
    \hline
    \textbf{Planifié} & \textbf{Progrès} & \textbf{Objectif} & \textbf{Responsable} \\\hline
      50\% & 25\% & Implémenter la lecture de courant à partir des capteurs de l'onduleur & Gabriel Q.\\\hline
      50\% & 25\% & Tests avec la première version de l'onduleur de l'équipe & Gabriel Q.\\\hline
      50\% & 10\% & Implémenter la lecture de la température à partir des capteurs de l'onduleur & Louis Tardif.\\\hline 
      25\% & 0\% & Implémenter les types de données pour les transformées, transformées inverses  & Louis Tardif.\\\hline
\end{tabularx}
\newline

\hfill \break
\textbf{\large Simulateur}
\\
\begin{tabularx}{\linewidth}{
    |>{\hsize=0.5\hsize}X|
    >{\hsize=0.5\hsize}X|
    >{\hsize=2.5\hsize}X|% 10% of 4\hsize 
    >{\hsize=0.5\hsize}X|% 30% of 4\hsize
       % sum=0.2\hsize for 4 columns
  }
    \hline
    \textbf{Planifié} & \textbf{Progrès} & \textbf{Objectif} & \textbf{Responsable} \\\hline
        100\% & 75\% & {Création et modification de tests unitaires} & Marian L-R \\\hline %28/03/2022
        100\% & 95\% & Afficher le bilan énergitique par système dans l'interface & Malik C.\\\hline
        75\% & 50\% & Comparaison entre deux simulation & Malik C.\\\hline
        100\% & 100\% & \st{Faire un algorithme de conversion entre la pédale et la vitesse} & Claude G.-P. \\\hline %28/03/2022
        100\% & 100\% & \st{Faire un test (CICD) pour s'assurer que le code fonctionne} & Claude G.-P. \\\hline %28/03/2022
        0\% & 0\% & Ajouter des fonctionnalités de contrôle de la simulation (choix du solver) & Claude G.-P. \\\hline %28/03/2022
        100\% & 85\% & Implémentation du générateur de parcours dans l'interface & William R.\\\hline
        0\% & 50\% & Version 1 du simulateur & Mathieu P.\\\hline
        100\% & 10\% & Implémentation algorithme génétique & Mathieu P.\\\hline


\end{tabularx}\\

\hfill \break
\textbf{\large Télémétrie/Instrumentation}\\
\begin{tabularx}{\linewidth}{
    |>{\hsize=0.5\hsize}X|
    >{\hsize=0.5\hsize}X|
    >{\hsize=2.5\hsize}X|% 10% of 4\hsize 
    >{\hsize=0.5\hsize}X|% 30% of 4\hsize
       % sum=0.2\hsize for 4 columns
  }
    \hline
    \textbf{Planifié} & \textbf{Progrès} & \textbf{Objectif} & \textbf{Responsable} \\\hline
     66 \% & 55\% & Construire une application cellulaire prête à envoyer des données du bluetooth et GPS vers un serveur & Alexandre B. \\\hline 
     100 \% & 90\% & Ajout des feux d'urgences dans les fonctions du STM32 & Alexandre B. \\\hline 
     0 \% & 5\% & Terminer l'implémentation bluetooth du STM32 en sortant l'information du CAN & Alexandre B. \\\hline 
     66 \% & 15\% & Afficher une consigne d'accélération selon la vitesse moyenne du véhicule sur le Epaper & Charles-E. G.\\\hline
     
\end{tabularx}

% TEMPLATE des lignes du tableau de taches accomplis cette semaine
%https://www.overleaf.com/project/60dbafb3c22aac53e265b6e6
% Tâche: résumé de la tâche
% Système : Numéro de système ou nom complet ex : Simulateur ou SIM1
% Responsable : Initiales ou nom complet ex : Gabriel Cabana ou G.C.
% Heures : Heures passées par le responsable à faire la tâche la semaine dernière (jeudi à jeudi)
%
%\textbf{Tâche} & \textbf{Système} & \textbf{Responsable} & \textbf{Heures}\\\hline
%    Tâche & Système & Responsable & Heures\\\hline
%    Tâche & Système & Responsable & Heures\\\hline
%    Tâche & Système & Responsable & Heures\\\hline
%    Tâche & Système & Responsable & Heures\\\hline
%    Tâche & Système & Responsable & Heures\\\hline
%    Tâche & Système & Responsable & Heures\\\hline
%    Tâche & Système & Responsable & Heures\\\hline
%    Tâche & Système & Responsable & Heures\\\hline
%    Tâche & Système & Responsable & Heures\\\hline
%    Tâche & Système & Responsable & Heures\\\hline
%    Tâche & Système & Responsable & Heures\\\hline
%    Tâche & Système & Responsable & Heures\\\hline
%  