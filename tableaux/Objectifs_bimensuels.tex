\textbf{\large Contrôle Moteur / BMS}\\
\begin{tabularx}{\linewidth}{
    |>{\hsize=0.5\hsize}X|
    >{\hsize=0.5\hsize}X|
    >{\hsize=2.5\hsize}X|% 10% of 4\hsize 
    >{\hsize=0.5\hsize}X|% 30% of 4\hsize
       % sum=0.2\hsize for 4 columns
  }
    \hline
    \textbf{Planifié} & \textbf{Progrès} & \textbf{Objectif} & \textbf{Responsable} \\\hline
      10\% & 5\% & Faire varier la vitesse d'un moteur avec une demande de torque & Gabriel Q.\\\hline
      0\% & 0\% & Tests avec la première version de l'onduleur de l'équipe & Gabriel Q.\\\hline
      75\% & 10\% & Faire l'étude de la consommation énergétique des STM32 & Louis Tardif.\\\hline 
\end{tabularx}
\newline

\hfill \break
\textbf{\large Simulateur}
\\
\begin{tabularx}{\linewidth}{
    |>{\hsize=0.5\hsize}X|
    >{\hsize=0.5\hsize}X|
    >{\hsize=2.5\hsize}X|% 10% of 4\hsize 
    >{\hsize=0.5\hsize}X|% 30% of 4\hsize
       % sum=0.2\hsize for 4 columns
  }
    \hline
    \textbf{Planifié} & \textbf{Progrès} & \textbf{Objectif} & \textbf{Responsable} \\\hline
        75\% & 75 \% & {Page d'accueil du simulateur} & Marian L-R.\\\hline
        25\% & 0 \% & {Création et modification de tests unitaires} & Marian L-R \\\hline
        100\% & 95\% & Afficher le bilan énergitique par système dans l'interface & Malik C.\\\hline
        75\% & 85\% & Comparaison entre deux simulation & Malik C.\\\hline
        100\% & 50\% & Implémentation du moteur & Claude G.-P. \\\hline
        0\% & 0\% & Ajouter des fonctionnalités de contrôle de la simulation (choix du solver) & Claude G.-P. \\\hline
        100\% & 50\% & Implémentation du générateur de parcours dans l'interface & William R.\\\hline
        100\% & 80\% & Stratégie de conduite & Mathieu P.\\\hline
        50\% & 10\% & Implémentation algorithme génétique & Mathieu P.\\\hline


\end{tabularx}\\

\hfill \break
\textbf{\large Télémétrie/Instrumentation}\\
\begin{tabularx}{\linewidth}{
    |>{\hsize=0.5\hsize}X|
    >{\hsize=0.5\hsize}X|
    >{\hsize=2.5\hsize}X|% 10% of 4\hsize 
    >{\hsize=0.5\hsize}X|% 30% of 4\hsize
       % sum=0.2\hsize for 4 columns
  }
    \hline
    \textbf{Planifié} & \textbf{Progrès} & \textbf{Objectif} & \textbf{Responsable} \\\hline
     100\% & 80\% & \st{Choisir si la LED LEDT10YGK2EE-CA produit un éclairage suffisant à la puissance désirée} & Alexandre B.\\\hline
     100\% & 100\% & \st{Démo avec un écran epaper ou autre low power pour afficher la vitesse} & Charles-E. G.\\\hline
     0 \% & 20\% & Construire une application cellulaire prête à envoyer des données du bluetooth et GPS vers un serveur & Alexandre B. \\\hline 
     0 \% & 0\% & Ajout des feux d'urgences dans les fonctions du STM32 & Alexandre B. \\\hline 
     0 \% & 0\% & Comminiquer avec BRP pour savoir quand un essuie-glace pourrait être reçu & Alexandre B. \\\hline 
     0 \% & 0\% & Afficher une consigne d'accélération selon la vitesse moyenne du véhicule sur le Epaper & Charles-E. G.\\\hline
     
\end{tabularx}

% TEMPLATE des lignes du tableau de taches accomplis cette semaine
%https://www.overleaf.com/project/60dbafb3c22aac53e265b6e6
% Tâche: résumé de la tâche
% Système : Numéro de système ou nom complet ex : Simulateur ou SIM1
% Responsable : Initiales ou nom complet ex : Gabriel Cabana ou G.C.
% Heures : Heures passées par le responsable à faire la tâche la semaine dernière (jeudi à jeudi)
%
%\textbf{Tâche} & \textbf{Système} & \textbf{Responsable} & \textbf{Heures}\\\hline
%    Tâche & Système & Responsable & Heures\\\hline
%    Tâche & Système & Responsable & Heures\\\hline
%    Tâche & Système & Responsable & Heures\\\hline
%    Tâche & Système & Responsable & Heures\\\hline
%    Tâche & Système & Responsable & Heures\\\hline
%    Tâche & Système & Responsable & Heures\\\hline
%    Tâche & Système & Responsable & Heures\\\hline
%    Tâche & Système & Responsable & Heures\\\hline
%    Tâche & Système & Responsable & Heures\\\hline
%    Tâche & Système & Responsable & Heures\\\hline
%    Tâche & Système & Responsable & Heures\\\hline
%    Tâche & Système & Responsable & Heures\\\hline
%  