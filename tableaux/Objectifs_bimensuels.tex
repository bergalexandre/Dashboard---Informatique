\textbf{\large Contrôle Moteur / BMS}\\
\begin{tabularx}{\linewidth}{
    |>{\hsize=0.5\hsize}X|
    >{\hsize=0.5\hsize}X|
    >{\hsize=2.5\hsize}X|% 10% of 4\hsize 
    >{\hsize=0.5\hsize}X|% 30% of 4\hsize
       % sum=0.2\hsize for 4 columns
  }
    \hline
    \textbf{Planifié} & \textbf{Progrès} & \textbf{Objectif} & \textbf{Responsable} \\\hline
      100\% & 20\% & Faire varier le duty cycle du SPWM en temps réel & Gabriel Q.\\\hline
      100\% & 85\% & Faire une communication Intrumentation - BMS & Louis T.\\\hline
      100\% & 0\% & Faire un banc de test simple avec un moteur ou une charge & Gabriel Q.\\\hline 
\end{tabularx}
\newline

\hfill \break
\textbf{\large Simulateur}
\\
\begin{tabularx}{\linewidth}{
    |>{\hsize=0.5\hsize}X|
    >{\hsize=0.5\hsize}X|
    >{\hsize=2.5\hsize}X|% 10% of 4\hsize 
    >{\hsize=0.5\hsize}X|% 30% of 4\hsize
       % sum=0.2\hsize for 4 columns
  }
    \hline
    \textbf{Planifié} & \textbf{Progrès} & \textbf{Objectif} & \textbf{Responsable} \\\hline
        100\% & 100\% & \st{Simuler déplacement du véhicule avec consigne pilote} & Claude G.-P.\\\hline
        50\% & 70\% & Simuler plusieurs cellules dans la batterie & Mathieu P.\\\hline
        75\% & 50 \% & Implémentation parcours pour le véhicule & William R.\\\hline 
        75\% & 50 \% & Implémentation de la première stratégie de conduite & Charles-Etienne G.\\\hline 
        100\% & 100\% & \st{Préparer un design review} & Claude G.-P.\\\hline
        50\% & 0 \% & Modification des paramètres clés par l'interface graphique & Malik C.\\\hline
        50\% & 0 \% & Intégration des premières formules de l'inverter & Marian Lambert-Rivest \\\hline
        100\% & 100\% & \st{Peaufiner la force latéral de virage dans la simulation} & Claude G.-P. \\\hline
        0\% & 80\% & Calculer le bilan énergétique par système  & Mathieu P.\\\hline
        0\% & 0\% & Ajouter des classes pour les paramètres clés & Claude G.-P. \\\hline
        0\% & 0\% & Débuter l'implémentation de la map moteur & Claude G.-P. \\\hline
\end{tabularx}\\

\hfill \break
\textbf{\large Télémétrie/Instrumentation}\\
\begin{tabularx}{\linewidth}{
    |>{\hsize=0.5\hsize}X|
    >{\hsize=0.5\hsize}X|
    >{\hsize=2.5\hsize}X|% 10% of 4\hsize 
    >{\hsize=0.5\hsize}X|% 30% of 4\hsize
       % sum=0.2\hsize for 4 columns
  }
    \hline
    \textbf{Planifié} & \textbf{Progrès} & \textbf{Objectif} & \textbf{Responsable} \\\hline
    50\% & 40\% & Simuler sur breadboard un tableau de bord à l'aide du stm32, leds et boutons &  Alexandre B. \\
     & & Extrant: Preuve de concepte avant de concevoir le circuit électronique de la cabine &  \\\hline
     0\% & 0\% & Choisir si la LED LEDT10YGK2EE-CA produit un éclairage suffisant à la puissance désirée & Alexandre B.\\\hline
     0\% & 0\% & Démo avec un écran epaper ou autre low power pour afficher la vitesse & grand G\\\hline
\end{tabularx}

% TEMPLATE des lignes du tableau de taches accomplis cette semaine
%https://www.overleaf.com/project/60dbafb3c22aac53e265b6e6
% Tâche: résumé de la tâche
% Système : Numéro de système ou nom complet ex : Simulateur ou SIM1
% Responsable : Initiales ou nom complet ex : Gabriel Cabana ou G.C.
% Heures : Heures passées par le responsable à faire la tâche la semaine dernière (jeudi à jeudi)
%
%\textbf{Tâche} & \textbf{Système} & \textbf{Responsable} & \textbf{Heures}\\\hline
%    Tâche & Système & Responsable & Heures\\\hline
%    Tâche & Système & Responsable & Heures\\\hline
%    Tâche & Système & Responsable & Heures\\\hline
%    Tâche & Système & Responsable & Heures\\\hline
%    Tâche & Système & Responsable & Heures\\\hline
%    Tâche & Système & Responsable & Heures\\\hline
%    Tâche & Système & Responsable & Heures\\\hline
%    Tâche & Système & Responsable & Heures\\\hline
%    Tâche & Système & Responsable & Heures\\\hline
%    Tâche & Système & Responsable & Heures\\\hline
%    Tâche & Système & Responsable & Heures\\\hline
%    Tâche & Système & Responsable & Heures\\\hline
%  