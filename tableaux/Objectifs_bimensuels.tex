\textbf{\large Contrôle Moteur / BMS}\\
\begin{tabularx}{\linewidth}{
    |>{\hsize=0.33\hsize}X|
    >{\hsize=0.33\hsize}X|
    >{\hsize=0.33\hsize}X|
    >{\hsize=2.5\hsize}X|% 10% of 4\hsize 
    >{\hsize=0.5\hsize}X|% 30% of 4\hsize
       % sum=0.2\hsize for 4 columns
  }
    \hline
    \textbf{Planifié} & \textbf{Progrès} & \textbf{Heures} &\textbf{Objectif} & \textbf{Responsable} \\\hline
    100\% & 100\% & 6 & Synchroniser les lectures de courants de phase aux PWMs du contrôle moteur & Gabriel Q.\\\hline
    50\% & 50\% & 4 & Tester l'encodeur avec le nouveau montage bien aligné & Gabriel Q.\\\hline
    100\% & 100\% & 10 & Intégrer la lecture de tension des capteurs de courant  & Louis Tardif.\\\hline
\end{tabularx}
\newline

\hfill \break
\textbf{\large Simulateur}
\\
\begin{tabularx}{\linewidth}{
    |>{\hsize=0.33\hsize}X|
    >{\hsize=0.33\hsize}X|
    >{\hsize=0.33\hsize}X|
    >{\hsize=2.5\hsize}X|% 10% of 4\hsize 
    >{\hsize=0.5\hsize}X|% 30% of 4\hsize
       % sum=0.2\hsize for 4 columns
  }
    \hline
    \textbf{Planifié} & \textbf{Progrès} & \textbf{Heures} &\textbf{Objectif} & \textbf{Responsable} \\\hline
        75\% & 75\% & 11 &  Format data pour logiciel Mtest-7 & Malik C.\\\hline
        0 \% & 0\% & 0 &  Intégration du parcours sur le dyno de Pascal Messier. & Malik C.\\\hline % nouveau stuff
        100\% & 100\% & 18 &  \st{Rendre la modification des paramêtres plus intuitif} & Mathieu P.\\\hline % Claude 2022-09-21
        75\% & 60\% & 12 &  Véhicule fait des arrêts (stratégie de conduite) & Claude G.-P.\\\hline % Claude 2022-09-21
        100\% & 75\% & 16 &  décélération/acélération dans les virages (stratégie de conduite) & Mathieu P. \\\hline % ?

\end{tabularx}\\

\hfill \break
\textbf{\large Instrumentation}\\
\begin{tabularx}{\linewidth}{
    |>{\hsize=0.33\hsize}X|
    >{\hsize=0.33\hsize}X|
    >{\hsize=0.33\hsize}X|
    >{\hsize=2.5\hsize}X|% 10% of 4\hsize 
    >{\hsize=0.5\hsize}X|% 30% of 4\hsize
       % sum=0.2\hsize for 4 columns
  }
    \hline
    \textbf{Planifié} & \textbf{Progrès} & \textbf{Heures} &\textbf{Objectif} & \textbf{Responsable} \\\hline
     50 \% & 100\% & 1 &   Plan du cablage d'instrumentation & Alexandre B. \\\hline 
     80 \% & 100\% & 18 &   Librairie d'affichage sur le E-Paper testée et intégrée & Charles-E. G. \\\hline 
     100 \% & 100\% & 8 &   \st{Communication entre l'application mobile et le simulateur (MQTT)} & Marian L.R. \\\hline 
     100 \% & 100\% & 2 &   \st{Mise à jour du code de l'essuie glace pour utiliser un servo moteur.} & Alexandre Bergeron \\\hline 
     30 \% & 50\% & 7 & Lecture de la pédale de torque. & Alexandre Bergeron \\\hline
     10 \% & 33\% & 0 & Création d'une librairie CAN. & William Rousseau \\\hline
\end{tabularx}

% TEMPLATE des lignes du tableau de taches accomplis cette semaine
%https://www.overleaf.com/project/60dbafb3c22aac53e265b6e6
% Tâche: résumé de la tâche
% Système : Numéro de système ou nom complet ex : Simulateur ou SIM1
% Responsable : Initiales ou nom complet ex : Gabriel Cabana ou G.C.
% Heures : Heures passées par le responsable à faire la tâche la semaine dernière (jeudi à jeudi)
%
%\textbf{Tâche} & \textbf{Système} & \textbf{Responsable} & \textbf{Heures}\\\hline
%    Tâche & Système & Responsable & Heures\\\hline
%    Tâche & Système & Responsable & Heures\\\hline
%    Tâche & Système & Responsable & Heures\\\hline
%    Tâche & Système & Responsable & Heures\\\hline
%    Tâche & Système & Responsable & Heures\\\hline
%    Tâche & Système & Responsable & Heures\\\hline
%    Tâche & Système & Responsable & Heures\\\hline
%    Tâche & Système & Responsable & Heures\\\hline
%    Tâche & Système & Responsable & Heures\\\hline
%    Tâche & Système & Responsable & Heures\\\hline
%    Tâche & Système & Responsable & Heures\\\hline
%    Tâche & Système & Responsable & Heures\\\hline
%  