\textbf{\large Contrôle Moteur / BMS}\\
\begin{tabularx}{\linewidth}{
    |>{\hsize=1.75\hsize}X|% 10% of 4\hsize 
    >{\hsize=0.25\hsize}X|% 30% of 4\hsize
       % sum=0.2\hsize for 4 columns
  }
    \hline
    \textbf{Objectif} & \textbf{Responsable} \\\hline
       Générer un SPWM simple complémentaire & Gabriel Q.\\\hline 
       Communiquer avec capteurs de température par SMBUS & Louis T.\\\hline
       Faire un banc de test simple avec un moteur ou une charge & Gabriel Q.\\\hline 
\end{tabularx}
\newline

\hfill \break
\textbf{\large Simulateur}
\\
\begin{tabularx}{\linewidth}{
    |>{\hsize=1.75\hsize}X|% 10% of 4\hsize 
    >{\hsize=0.25\hsize}X|% 30% of 4\hsize
       % sum=0.2\hsize for 4 columns
  }
    \hline
    \textbf{Objectif} & \textbf{Responsable} \\\hline
        \st{Statuer sur l’implémentation de l’inverter} & Marian L.-R.\\\hline  
        Implémentation dynamique latérale ettransformation des mouvements & Claude G.-P.\\\hline
        Implémentation des formules de la transmition & Mathieu P.\\\hline 
        Implémentation parcours pour le véhicule & Charles-Étienne G.\\\hline 
        Implémenter le contrôle de la simulation via l’interface graphique & William R.\\\hline 
        \st{Implémenter les tests sur les équations déjà en place} & Malik C.\\\hline 
\end{tabularx}\\

\hfill \break
\textbf{\large Télémétrie/Instrumentation}\\
\begin{tabularx}{\linewidth}{
    |>{\hsize=1.75\hsize}X|% 10% of 4\hsize 
    >{\hsize=0.25\hsize}X|% 30% of 4\hsize
       % sum=0.2\hsize for 4 columns
  }
    \hline
    \textbf{Objectif} & \textbf{Responsable} \\\hline
     Accomplir une démo bluetooth  & Alexandre B. \\\hline 
     Investiguer sur les besoins matériels et logiciel pour les lumières et essuie glace  & Alexandre B. \\\hline
     Inverstiguer sur les besoins matériels et logiciel pour lecture de la pédale de freins  & Charles-Étienne G.\\\hline 
\end{tabularx}

% TEMPLATE des lignes du tableau de taches accomplis cette semaine
%https://www.overleaf.com/project/60dbafb3c22aac53e265b6e6
% Tâche: résumé de la tâche
% Système : Numéro de système ou nom complet ex : Simulateur ou SIM1
% Responsable : Initiales ou nom complet ex : Gabriel Cabana ou G.C.
% Heures : Heures passées par le responsable à faire la tâche la semaine dernière (jeudi à jeudi)
%
%\textbf{Tâche} & \textbf{Système} & \textbf{Responsable} & \textbf{Heures}\\\hline
%    Tâche & Système & Responsable & Heures\\\hline
%    Tâche & Système & Responsable & Heures\\\hline
%    Tâche & Système & Responsable & Heures\\\hline
%    Tâche & Système & Responsable & Heures\\\hline
%    Tâche & Système & Responsable & Heures\\\hline
%    Tâche & Système & Responsable & Heures\\\hline
%    Tâche & Système & Responsable & Heures\\\hline
%    Tâche & Système & Responsable & Heures\\\hline
%    Tâche & Système & Responsable & Heures\\\hline
%    Tâche & Système & Responsable & Heures\\\hline
%    Tâche & Système & Responsable & Heures\\\hline
%    Tâche & Système & Responsable & Heures\\\hline
%  