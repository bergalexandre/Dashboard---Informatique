\textbf{\large Contrôle Moteur / BMS}\\
\begin{tabularx}{\linewidth}{
    |>{\hsize=0.33\hsize}X|
    >{\hsize=0.33\hsize}X|
    >{\hsize=0.33\hsize}X|
    >{\hsize=2.5\hsize}X|% 10% of 4\hsize 
    >{\hsize=0.5\hsize}X|% 30% of 4\hsize
       % sum=0.2\hsize for 4 columns
  }
    \hline
    \textbf{Planifié(\%)} & \textbf{Progrès(\%)} & \textbf{Planifié(H)} &\textbf{Objectif} & \textbf{Responsable} \\\hline
    100\% & 100\% & 30 & \st{ Refactor en correction des calculs FOC} & Louis T.\\\hline
    50\% & 50\% & 40 & Tester la prise de mesures et le "tuning" avec STM32CubeMonitor & Louis T.\\\hline
    50\% & 50\% & 10 & Avoir un code qui permet de tester le moteur sur le dynamomètre & Gabriel Q.\\\hline
    100\% & 100\% & 5 & Connaitre le RPM du moteur & Gabriel Q.\\\hline
    50\% & 50\% & 15 & Avoir une 'dead-zone' ou une transition open->closed loop pour le contrôle  & Louis T.\\\hline
    10\% & 10\% & 5 & Contrôler la direction de rotation du moteur (Horaire, anti-horaire)  & Louis T.\\\hline
\end{tabularx}
\newline

\hfill \break
\textbf{\large Simulateur}
\\
\begin{tabularx}{\linewidth}{
    |>{\hsize=0.33\hsize}X|
    >{\hsize=0.33\hsize}X|
    >{\hsize=0.33\hsize}X|
    >{\hsize=2.5\hsize}X|% 10% of 4\hsize 
    >{\hsize=0.5\hsize}X|% 30% of 4\hsize 
       % sum=0.2\hsize for 4 columns 
  }
    \hline
    \textbf{Planifié(\%)} & \textbf{Progrès(\%)} & \textbf{Planifié(H)} &\textbf{Objectif} & \textbf{Responsable} \\\hline
        75 \% & 70\% & 12 &  Création script pour sortir torque en fonction RPM de datasheet. & Malik C.\\\hline
        0 \% & 0\% & 6 &  Intégration du parcours sur le dyno de Pascal Messier. & Malik C.\\\hline % nouveau stuff
        100\% & 95\% & 12 &  Véhicule fait des arrêts (stratégie de conduite) & Claude G.-P.\\\hline % Claude 2022-09-28
        100\% & 95\% & 16 &  Décélération/acélération dans les virages (stratégie de conduite) & Mathieu P. \\\hline % ?
        25\% & 25\% & 30 &  Conception contrôleur de vitesse (machine à état/logique floue)  & Mathieu P. \\\hline % ?

\end{tabularx}\\

\hfill \break
\textbf{\large Instrumentation}\\
\begin{tabularx}{\linewidth}{
    |>{\hsize=0.33\hsize}X|
    >{\hsize=0.33\hsize}X|
    >{\hsize=0.33\hsize}X|
    >{\hsize=2.5\hsize}X|% 10% of 4\hsize 
    >{\hsize=0.5\hsize}X|% 30% of 4\hsize
       % sum=0.2\hsize for 4 columns
  }
    \hline
    \textbf{Planifié(\%)} & \textbf{Progrès(\%)} & \textbf{Planifié(H)} &\textbf{Objectif} & \textbf{Responsable} \\\hline
     100 \% & 100\% & 1 &   \st{Plan du cablage d'instrumentation} & Alexandre B. \\\hline 
     75 \% & 0\% & 9 &  Contrôle de l'intensité des phares avant & Charles-E. G. \\\hline 
     100 \% & 100\% & 7 & \st{Lecture de la pédale de torque.} & Alexandre B. \\\hline
     75 \% & 50\% & 8 & Création d'une librairie CAN. & William R. et Marian L.R. \\\hline
     40 \% & 40\% & 24 & Test de la carte prototype & Alexandre B. et Charles-E. G. \\\hline
\end{tabularx}

% TEMPLATE des lignes du tableau de taches accomplis cette semaine
%https://www.overleaf.com/project/60dbafb3c22aac53e265b6e6
% Tâche: résumé de la tâche
% Système : Numéro de système ou nom complet ex : Simulateur ou SIM1
% Responsable : Initiales ou nom complet ex : Gabriel Cabana ou G.C.
% Heures : Heures passées par le responsable à faire la tâche la semaine dernière (jeudi à jeudi)
%
%\textbf{Tâche} & \textbf{Système} & \textbf{Responsable} & \textbf{Heures}\\\hline
%    Tâche & Système & Responsable & Heures\\\hline
%    Tâche & Système & Responsable & Heures\\\hline
%    Tâche & Système & Responsable & Heures\\\hline
%    Tâche & Système & Responsable & Heures\\\hline
%    Tâche & Système & Responsable & Heures\\\hline
%    Tâche & Système & Responsable & Heures\\\hline
%    Tâche & Système & Responsable & Heures\\\hline
%    Tâche & Système & Responsable & Heures\\\hline
%    Tâche & Système & Responsable & Heures\\\hline
%    Tâche & Système & Responsable & Heures\\\hline
%    Tâche & Système & Responsable & Heures\\\hline
%    Tâche & Système & Responsable & Heures\\\hline
%  