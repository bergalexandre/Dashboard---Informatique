\textbf{\large Contrôle Moteur / BMS}\\
\begin{tabularx}{\linewidth}{
    |>{\hsize=0.12\hsize}X|
    >{\hsize=1.63\hsize}X|% 10% of 4\hsize 
    >{\hsize=0.25\hsize}X|% 30% of 4\hsize
       % sum=0.2\hsize for 4 columns
  }
    \hline
    \textbf{Progrès} & \textbf{Objectif} & \textbf{Responsable} \\\hline
      100\% & \st{Générer un SPWM simple complémentaire} & Gabriel Q.\\\hline
      0\% & Faire varier le duty cycle du SPWM en temps réel & Gabriel Q.\\\hline
      100\% & \st{Communiquer avec capteurs de température par SMBUS} & Louis T.\\\hline
      0\% & Faire un banc de test simple avec un moteur ou une charge & Gabriel Q.\\\hline 
\end{tabularx}
\newline

\hfill \break
\textbf{\large Simulateur}
\\
\begin{tabularx}{\linewidth}{
    |>{\hsize=0.12\hsize}X|
    >{\hsize=1.63\hsize}X|% 10% of 4\hsize 
    >{\hsize=0.25\hsize}X|% 30% of 4\hsize
       % sum=0.2\hsize for 4 columns
  }
    \hline
    \textbf{Progrès} & \textbf{Objectif} & \textbf{Responsable} \\\hline
        100\% & \st{Implémentation formule dynamique latérale et transformation des mouvements} & Claude G.-P.\\\hline
        \% & Simuler déplacement du véhicule avec consigne pilote & Claude G.-P.\\\hline
        100\% & \st{Implémentation des formules de la transmition} & Mathieu P.\\\hline
        \% & Simuler plusieurs cellules dans la batterie & Mathieu P.\\\hline
        \% & Implémentation parcours pour le véhicule & Charles-Étienne G.\\\hline 
        100\% & \st{Implémenter le contrôle de la simulation via l’interface graphique} & William R.\\\hline 
        100 \% & \st{Implémenter les tests sur les équations déjà en place} & Malik C.\\\hline 
        0 \% & Préparer un design review & Claude G.-P.\\\hline
        0 \% & Modification des paramètres clés par l'interface graphique & Malik C.\\\hline
        0 \% & Intégration des premières formules de l'inverter & Marian Lambert-Rivest \\\hline
        0 \% & Peaufiner la force latéral de virage dans la simulation & Claude G.-P. \\\hline
        0 \% &  & \\\hline
        0 \% &  & \\\hline

\end{tabularx}\\

\hfill \break
\textbf{\large Télémétrie/Instrumentation}\\
\begin{tabularx}{\linewidth}{
    |>{\hsize=0.12\hsize}X|
    >{\hsize=1.63\hsize}X|% 10% of 4\hsize 
    >{\hsize=0.25\hsize}X|% 30% of 4\hsize
       % sum=0.2\hsize for 4 columns
  }
    \hline
    \textbf{Progrès} & \textbf{Objectif} & \textbf{Responsable} \\\hline
      100\% & \st{Accomplir une démo bluetooth}  & Alexandre B. \\\hline 
     50\% & \st{Investiguer sur les besoins matériels et logiciel pour les lumières et essuie glace}  & Alexandre B. \\\hline
     50\% & \st{Inverstiguer sur les besoins matériels et logiciel pour lecture de la pédale de freins}  & Charles-Étienne G.\\\hline
     0\% & Simuler sur breadboard un tableau de bord à l'aide du stm32, leds et boutons &  Alexandre B. \\
     & Extrant: Preuve de concepte avant de concevoir le circuit électronique de la cabine &  \\\hline
\end{tabularx}

% TEMPLATE des lignes du tableau de taches accomplis cette semaine
%https://www.overleaf.com/project/60dbafb3c22aac53e265b6e6
% Tâche: résumé de la tâche
% Système : Numéro de système ou nom complet ex : Simulateur ou SIM1
% Responsable : Initiales ou nom complet ex : Gabriel Cabana ou G.C.
% Heures : Heures passées par le responsable à faire la tâche la semaine dernière (jeudi à jeudi)
%
%\textbf{Tâche} & \textbf{Système} & \textbf{Responsable} & \textbf{Heures}\\\hline
%    Tâche & Système & Responsable & Heures\\\hline
%    Tâche & Système & Responsable & Heures\\\hline
%    Tâche & Système & Responsable & Heures\\\hline
%    Tâche & Système & Responsable & Heures\\\hline
%    Tâche & Système & Responsable & Heures\\\hline
%    Tâche & Système & Responsable & Heures\\\hline
%    Tâche & Système & Responsable & Heures\\\hline
%    Tâche & Système & Responsable & Heures\\\hline
%    Tâche & Système & Responsable & Heures\\\hline
%    Tâche & Système & Responsable & Heures\\\hline
%    Tâche & Système & Responsable & Heures\\\hline
%    Tâche & Système & Responsable & Heures\\\hline
%  