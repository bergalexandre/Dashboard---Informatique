\begin{tabularx}{\linewidth}{
    |>{\hsize=1.5\hsize}X|% 10% of 4\hsize 
    >{\hsize=0.5\hsize}X|% 30% of 4\hsize
    >{\hsize=0.5\hsize}X|% 30% of 4\hsize
    >{\hsize=0.5\hsize}X|% 30% of 4\hsize 
       % sum=0.2\hsize for 4 columns
  }
    \hline
    \textbf{Problème} & \textbf{État} & \textbf{Système} & \textbf{Responsable}\\\hline
   Les largeurs d'impulsion sont un peu plus large de ce qui a été prévu pour le contrôle de l'essuie glace. & En évaluation & Instrumentation & Alexandre Bergeron \\\hline %2022-04-08
  \end{tabularx}
    
    
 % Template des problèmes rencontés:
 %
 % Problème : le problème rencontré cette semaine
 % Système  : le nom ou numéro système que le problème touche ex: Simulateur ou SIM1
 % Responsable : Nom ou Initiales du ou des personnes touchées par ce problèmes ex: G.C. C.E.G.
 %
 %  \hline
 %  \textbf{Problème} & \textbf{Système} & \textbf{Responsable}\\\hline
 %  Problème & Système & Responsable \\\hline
 %  Problème & Système & Responsable \\\hline
 %  Problème & Système & Responsable \\\hline
 %  Problème & Système & Responsable \\\hline
 %  Problème & Système & Responsable \\\hline
 %  Problème & Système & Responsable \\\hline
 %  Problème & Système & Responsable \\\hline
 %